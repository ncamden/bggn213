% Options for packages loaded elsewhere
\PassOptionsToPackage{unicode}{hyperref}
\PassOptionsToPackage{hyphens}{url}
\PassOptionsToPackage{dvipsnames,svgnames,x11names}{xcolor}
%
\documentclass[
  letterpaper,
  DIV=11,
  numbers=noendperiod]{scrartcl}

\usepackage{amsmath,amssymb}
\usepackage{lmodern}
\usepackage{iftex}
\ifPDFTeX
  \usepackage[T1]{fontenc}
  \usepackage[utf8]{inputenc}
  \usepackage{textcomp} % provide euro and other symbols
\else % if luatex or xetex
  \usepackage{unicode-math}
  \defaultfontfeatures{Scale=MatchLowercase}
  \defaultfontfeatures[\rmfamily]{Ligatures=TeX,Scale=1}
\fi
% Use upquote if available, for straight quotes in verbatim environments
\IfFileExists{upquote.sty}{\usepackage{upquote}}{}
\IfFileExists{microtype.sty}{% use microtype if available
  \usepackage[]{microtype}
  \UseMicrotypeSet[protrusion]{basicmath} % disable protrusion for tt fonts
}{}
\makeatletter
\@ifundefined{KOMAClassName}{% if non-KOMA class
  \IfFileExists{parskip.sty}{%
    \usepackage{parskip}
  }{% else
    \setlength{\parindent}{0pt}
    \setlength{\parskip}{6pt plus 2pt minus 1pt}}
}{% if KOMA class
  \KOMAoptions{parskip=half}}
\makeatother
\usepackage{xcolor}
\setlength{\emergencystretch}{3em} % prevent overfull lines
\setcounter{secnumdepth}{-\maxdimen} % remove section numbering
% Make \paragraph and \subparagraph free-standing
\ifx\paragraph\undefined\else
  \let\oldparagraph\paragraph
  \renewcommand{\paragraph}[1]{\oldparagraph{#1}\mbox{}}
\fi
\ifx\subparagraph\undefined\else
  \let\oldsubparagraph\subparagraph
  \renewcommand{\subparagraph}[1]{\oldsubparagraph{#1}\mbox{}}
\fi

\usepackage{color}
\usepackage{fancyvrb}
\newcommand{\VerbBar}{|}
\newcommand{\VERB}{\Verb[commandchars=\\\{\}]}
\DefineVerbatimEnvironment{Highlighting}{Verbatim}{commandchars=\\\{\}}
% Add ',fontsize=\small' for more characters per line
\usepackage{framed}
\definecolor{shadecolor}{RGB}{241,243,245}
\newenvironment{Shaded}{\begin{snugshade}}{\end{snugshade}}
\newcommand{\AlertTok}[1]{\textcolor[rgb]{0.68,0.00,0.00}{#1}}
\newcommand{\AnnotationTok}[1]{\textcolor[rgb]{0.37,0.37,0.37}{#1}}
\newcommand{\AttributeTok}[1]{\textcolor[rgb]{0.40,0.45,0.13}{#1}}
\newcommand{\BaseNTok}[1]{\textcolor[rgb]{0.68,0.00,0.00}{#1}}
\newcommand{\BuiltInTok}[1]{\textcolor[rgb]{0.00,0.23,0.31}{#1}}
\newcommand{\CharTok}[1]{\textcolor[rgb]{0.13,0.47,0.30}{#1}}
\newcommand{\CommentTok}[1]{\textcolor[rgb]{0.37,0.37,0.37}{#1}}
\newcommand{\CommentVarTok}[1]{\textcolor[rgb]{0.37,0.37,0.37}{\textit{#1}}}
\newcommand{\ConstantTok}[1]{\textcolor[rgb]{0.56,0.35,0.01}{#1}}
\newcommand{\ControlFlowTok}[1]{\textcolor[rgb]{0.00,0.23,0.31}{#1}}
\newcommand{\DataTypeTok}[1]{\textcolor[rgb]{0.68,0.00,0.00}{#1}}
\newcommand{\DecValTok}[1]{\textcolor[rgb]{0.68,0.00,0.00}{#1}}
\newcommand{\DocumentationTok}[1]{\textcolor[rgb]{0.37,0.37,0.37}{\textit{#1}}}
\newcommand{\ErrorTok}[1]{\textcolor[rgb]{0.68,0.00,0.00}{#1}}
\newcommand{\ExtensionTok}[1]{\textcolor[rgb]{0.00,0.23,0.31}{#1}}
\newcommand{\FloatTok}[1]{\textcolor[rgb]{0.68,0.00,0.00}{#1}}
\newcommand{\FunctionTok}[1]{\textcolor[rgb]{0.28,0.35,0.67}{#1}}
\newcommand{\ImportTok}[1]{\textcolor[rgb]{0.00,0.46,0.62}{#1}}
\newcommand{\InformationTok}[1]{\textcolor[rgb]{0.37,0.37,0.37}{#1}}
\newcommand{\KeywordTok}[1]{\textcolor[rgb]{0.00,0.23,0.31}{#1}}
\newcommand{\NormalTok}[1]{\textcolor[rgb]{0.00,0.23,0.31}{#1}}
\newcommand{\OperatorTok}[1]{\textcolor[rgb]{0.37,0.37,0.37}{#1}}
\newcommand{\OtherTok}[1]{\textcolor[rgb]{0.00,0.23,0.31}{#1}}
\newcommand{\PreprocessorTok}[1]{\textcolor[rgb]{0.68,0.00,0.00}{#1}}
\newcommand{\RegionMarkerTok}[1]{\textcolor[rgb]{0.00,0.23,0.31}{#1}}
\newcommand{\SpecialCharTok}[1]{\textcolor[rgb]{0.37,0.37,0.37}{#1}}
\newcommand{\SpecialStringTok}[1]{\textcolor[rgb]{0.13,0.47,0.30}{#1}}
\newcommand{\StringTok}[1]{\textcolor[rgb]{0.13,0.47,0.30}{#1}}
\newcommand{\VariableTok}[1]{\textcolor[rgb]{0.07,0.07,0.07}{#1}}
\newcommand{\VerbatimStringTok}[1]{\textcolor[rgb]{0.13,0.47,0.30}{#1}}
\newcommand{\WarningTok}[1]{\textcolor[rgb]{0.37,0.37,0.37}{\textit{#1}}}

\providecommand{\tightlist}{%
  \setlength{\itemsep}{0pt}\setlength{\parskip}{0pt}}\usepackage{longtable,booktabs,array}
\usepackage{calc} % for calculating minipage widths
% Correct order of tables after \paragraph or \subparagraph
\usepackage{etoolbox}
\makeatletter
\patchcmd\longtable{\par}{\if@noskipsec\mbox{}\fi\par}{}{}
\makeatother
% Allow footnotes in longtable head/foot
\IfFileExists{footnotehyper.sty}{\usepackage{footnotehyper}}{\usepackage{footnote}}
\makesavenoteenv{longtable}
\usepackage{graphicx}
\makeatletter
\def\maxwidth{\ifdim\Gin@nat@width>\linewidth\linewidth\else\Gin@nat@width\fi}
\def\maxheight{\ifdim\Gin@nat@height>\textheight\textheight\else\Gin@nat@height\fi}
\makeatother
% Scale images if necessary, so that they will not overflow the page
% margins by default, and it is still possible to overwrite the defaults
% using explicit options in \includegraphics[width, height, ...]{}
\setkeys{Gin}{width=\maxwidth,height=\maxheight,keepaspectratio}
% Set default figure placement to htbp
\makeatletter
\def\fps@figure{htbp}
\makeatother

\KOMAoption{captions}{tableheading}
\makeatletter
\makeatother
\makeatletter
\makeatother
\makeatletter
\@ifpackageloaded{caption}{}{\usepackage{caption}}
\AtBeginDocument{%
\ifdefined\contentsname
  \renewcommand*\contentsname{Table of contents}
\else
  \newcommand\contentsname{Table of contents}
\fi
\ifdefined\listfigurename
  \renewcommand*\listfigurename{List of Figures}
\else
  \newcommand\listfigurename{List of Figures}
\fi
\ifdefined\listtablename
  \renewcommand*\listtablename{List of Tables}
\else
  \newcommand\listtablename{List of Tables}
\fi
\ifdefined\figurename
  \renewcommand*\figurename{Figure}
\else
  \newcommand\figurename{Figure}
\fi
\ifdefined\tablename
  \renewcommand*\tablename{Table}
\else
  \newcommand\tablename{Table}
\fi
}
\@ifpackageloaded{float}{}{\usepackage{float}}
\floatstyle{ruled}
\@ifundefined{c@chapter}{\newfloat{codelisting}{h}{lop}}{\newfloat{codelisting}{h}{lop}[chapter]}
\floatname{codelisting}{Listing}
\newcommand*\listoflistings{\listof{codelisting}{List of Listings}}
\makeatother
\makeatletter
\@ifpackageloaded{caption}{}{\usepackage{caption}}
\@ifpackageloaded{subcaption}{}{\usepackage{subcaption}}
\makeatother
\makeatletter
\@ifpackageloaded{tcolorbox}{}{\usepackage[many]{tcolorbox}}
\makeatother
\makeatletter
\@ifundefined{shadecolor}{\definecolor{shadecolor}{rgb}{.97, .97, .97}}
\makeatother
\makeatletter
\makeatother
\ifLuaTeX
  \usepackage{selnolig}  % disable illegal ligatures
\fi
\IfFileExists{bookmark.sty}{\usepackage{bookmark}}{\usepackage{hyperref}}
\IfFileExists{xurl.sty}{\usepackage{xurl}}{} % add URL line breaks if available
\urlstyle{same} % disable monospaced font for URLs
\hypersetup{
  pdftitle={Class06},
  pdfauthor={Nichelle Camden},
  colorlinks=true,
  linkcolor={blue},
  filecolor={Maroon},
  citecolor={Blue},
  urlcolor={Blue},
  pdfcreator={LaTeX via pandoc}}

\title{Class06}
\author{Nichelle Camden}
\date{2022-10-14}

\begin{document}
\maketitle
\ifdefined\Shaded\renewenvironment{Shaded}{\begin{tcolorbox}[frame hidden, borderline west={3pt}{0pt}{shadecolor}, enhanced, boxrule=0pt, breakable, sharp corners, interior hidden]}{\end{tcolorbox}}\fi

\renewcommand*\contentsname{Table of contents}
{
\hypersetup{linkcolor=}
\setcounter{tocdepth}{3}
\tableofcontents
}
\hypertarget{function-basics}{%
\subsection{Function basics}\label{function-basics}}

All functions in R have at least 3 things:

\begin{itemize}
\tightlist
\item
  A \textbf{name} (we pick this),
\item
  Input \textbf{arguments} (there can be loads that are comma
  separated),
\item
  A \textbf{body} (the R code that does the work).
\end{itemize}

\begin{Shaded}
\begin{Highlighting}[]
\CommentTok{\# example input vectors to start with }
\NormalTok{student1 }\OtherTok{\textless{}{-}} \FunctionTok{c}\NormalTok{(}\DecValTok{100}\NormalTok{, }\DecValTok{100}\NormalTok{, }\DecValTok{100}\NormalTok{, }\DecValTok{100}\NormalTok{, }\DecValTok{100}\NormalTok{, }\DecValTok{100}\NormalTok{, }\DecValTok{100}\NormalTok{, }\DecValTok{90}\NormalTok{)}
\NormalTok{student2 }\OtherTok{\textless{}{-}} \FunctionTok{c}\NormalTok{(}\DecValTok{100}\NormalTok{, }\ConstantTok{NA}\NormalTok{, }\DecValTok{90}\NormalTok{, }\DecValTok{90}\NormalTok{, }\DecValTok{90}\NormalTok{, }\DecValTok{90}\NormalTok{, }\DecValTok{97}\NormalTok{, }\DecValTok{80}\NormalTok{)}
\NormalTok{student3 }\OtherTok{\textless{}{-}} \FunctionTok{c}\NormalTok{(}\DecValTok{90}\NormalTok{, }\ConstantTok{NA}\NormalTok{, }\ConstantTok{NA}\NormalTok{, }\ConstantTok{NA}\NormalTok{, }\ConstantTok{NA}\NormalTok{, }\ConstantTok{NA}\NormalTok{, }\ConstantTok{NA}\NormalTok{, }\ConstantTok{NA}\NormalTok{)}

\CommentTok{\#and later we\textquotesingle{}ll try it with this data}
\NormalTok{gradebook }\OtherTok{\textless{}{-}} \FunctionTok{read.csv}\NormalTok{(}\StringTok{"student\_homework.csv"}\NormalTok{)}
\end{Highlighting}
\end{Shaded}

I can use the \texttt{mean()} function to get the average

\begin{Shaded}
\begin{Highlighting}[]
\FunctionTok{mean}\NormalTok{(student1)}
\end{Highlighting}
\end{Shaded}

\begin{verbatim}
[1] 98.75
\end{verbatim}

To find the lowest value, I can use the \texttt{min()} function

\begin{Shaded}
\begin{Highlighting}[]
\NormalTok{student1}
\end{Highlighting}
\end{Shaded}

\begin{verbatim}
[1] 100 100 100 100 100 100 100  90
\end{verbatim}

\begin{Shaded}
\begin{Highlighting}[]
\FunctionTok{min}\NormalTok{(student1)}
\end{Highlighting}
\end{Shaded}

\begin{verbatim}
[1] 90
\end{verbatim}

I found the \texttt{which.min()} function, what does it do?

\begin{Shaded}
\begin{Highlighting}[]
\FunctionTok{which.min}\NormalTok{(student1)}
\end{Highlighting}
\end{Shaded}

\begin{verbatim}
[1] 8
\end{verbatim}

The minimum value is in the 8th position for ``student1''.

There are a few different ways to take that value out from the average
calculation, but the index trick might be the easiest/most useful. (It
could also work to sort the students' scores from lowest to highest, and
then drop the first score for all.)

\begin{Shaded}
\begin{Highlighting}[]
\CommentTok{\#lil refresh on the index trick:}
\NormalTok{student1[}\SpecialCharTok{{-}}\DecValTok{8}\NormalTok{]}
\end{Highlighting}
\end{Shaded}

\begin{verbatim}
[1] 100 100 100 100 100 100 100
\end{verbatim}

\begin{Shaded}
\begin{Highlighting}[]
\NormalTok{student1[}\SpecialCharTok{{-}}\FunctionTok{which.min}\NormalTok{(student1)]}
\end{Highlighting}
\end{Shaded}

\begin{verbatim}
[1] 100 100 100 100 100 100 100
\end{verbatim}

Then I can take the mean

\begin{Shaded}
\begin{Highlighting}[]
\FunctionTok{mean}\NormalTok{(student1[}\SpecialCharTok{{-}}\FunctionTok{which.min}\NormalTok{(student1)])}
\end{Highlighting}
\end{Shaded}

\begin{verbatim}
[1] 100
\end{verbatim}

Will it work with ``student2''?

\begin{Shaded}
\begin{Highlighting}[]
\NormalTok{student2[}\SpecialCharTok{{-}}\FunctionTok{which.min}\NormalTok{(student2)]}
\end{Highlighting}
\end{Shaded}

\begin{verbatim}
[1] 100  NA  90  90  90  90  97
\end{verbatim}

Kind of, but not really. It just got rid of the lowest numerical value,
but not the actual lowest value (the NA)

\begin{Shaded}
\begin{Highlighting}[]
\FunctionTok{mean}\NormalTok{(student2[}\SpecialCharTok{{-}}\FunctionTok{which.min}\NormalTok{(student2)])}
\end{Highlighting}
\end{Shaded}

\begin{verbatim}
[1] NA
\end{verbatim}

\begin{Shaded}
\begin{Highlighting}[]
\FunctionTok{mean}\NormalTok{(student2, }\AttributeTok{na.rm=}\NormalTok{T)}
\end{Highlighting}
\end{Shaded}

\begin{verbatim}
[1] 91
\end{verbatim}

\hypertarget{we-need-another-way}{%
\section{We need another way\ldots{}}\label{we-need-another-way}}

Can I replace NA values with zero? No homework submission = 0 try google

\begin{Shaded}
\begin{Highlighting}[]
\FunctionTok{is.na}\NormalTok{(student2)}
\end{Highlighting}
\end{Shaded}

\begin{verbatim}
[1] FALSE  TRUE FALSE FALSE FALSE FALSE FALSE FALSE
\end{verbatim}

\begin{Shaded}
\begin{Highlighting}[]
\NormalTok{student2[ }\FunctionTok{is.na}\NormalTok{(student2) ] }\OtherTok{\textless{}{-}} \DecValTok{0}
\NormalTok{student2}
\end{Highlighting}
\end{Shaded}

\begin{verbatim}
[1] 100   0  90  90  90  90  97  80
\end{verbatim}

\begin{Shaded}
\begin{Highlighting}[]
\FunctionTok{c}\NormalTok{(T,T,F)}
\end{Highlighting}
\end{Shaded}

\begin{verbatim}
[1]  TRUE  TRUE FALSE
\end{verbatim}

\begin{Shaded}
\begin{Highlighting}[]
\SpecialCharTok{!}\FunctionTok{c}\NormalTok{(T,T,F)}
\end{Highlighting}
\end{Shaded}

\begin{verbatim}
[1] FALSE FALSE  TRUE
\end{verbatim}

\begin{Shaded}
\begin{Highlighting}[]
\CommentTok{\# the ! flips it}
\end{Highlighting}
\end{Shaded}

\begin{Shaded}
\begin{Highlighting}[]
\FunctionTok{is.na}\NormalTok{(student3)}
\end{Highlighting}
\end{Shaded}

\begin{verbatim}
[1] FALSE  TRUE  TRUE  TRUE  TRUE  TRUE  TRUE  TRUE
\end{verbatim}

\begin{Shaded}
\begin{Highlighting}[]
\NormalTok{student3[ }\FunctionTok{is.na}\NormalTok{(student3) ] }\OtherTok{\textless{}{-}} \DecValTok{0}
\NormalTok{student3}
\end{Highlighting}
\end{Shaded}

\begin{verbatim}
[1] 90  0  0  0  0  0  0  0
\end{verbatim}

\begin{Shaded}
\begin{Highlighting}[]
\NormalTok{positions }\OtherTok{\textless{}{-}} \FunctionTok{is.na}\NormalTok{(student2)}
\NormalTok{student2}
\end{Highlighting}
\end{Shaded}

\begin{verbatim}
[1] 100   0  90  90  90  90  97  80
\end{verbatim}

\begin{Shaded}
\begin{Highlighting}[]
\NormalTok{student3[positions] }\OtherTok{\textless{}{-}} \DecValTok{0}
\NormalTok{student3}
\end{Highlighting}
\end{Shaded}

\begin{verbatim}
[1] 90  0  0  0  0  0  0  0
\end{verbatim}

\begin{Shaded}
\begin{Highlighting}[]
\CommentTok{\#could also use "missing" instead of "positions"}
\end{Highlighting}
\end{Shaded}

\begin{Shaded}
\begin{Highlighting}[]
\NormalTok{student2[}\FunctionTok{is.na}\NormalTok{(student2)] }\OtherTok{\textless{}{-}} \DecValTok{0}
\FunctionTok{mean}\NormalTok{(student2 [}\SpecialCharTok{{-}}\FunctionTok{which.min}\NormalTok{(student2)])}
\end{Highlighting}
\end{Shaded}

\begin{verbatim}
[1] 91
\end{verbatim}

Re-write my snippet to be more simple for \textbf{Q1}

\hypertarget{q1}{%
\subsection{\texorpdfstring{\textbf{Q1}}{Q1}}\label{q1}}

\begin{Shaded}
\begin{Highlighting}[]
\NormalTok{x }\OtherTok{\textless{}{-}}\NormalTok{ student1}
\NormalTok{x[}\FunctionTok{is.na}\NormalTok{(x)] }\OtherTok{\textless{}{-}} \DecValTok{0}
\FunctionTok{mean}\NormalTok{(x[}\SpecialCharTok{{-}}\FunctionTok{which.min}\NormalTok{(x)])}
\end{Highlighting}
\end{Shaded}

\begin{verbatim}
[1] 100
\end{verbatim}

\begin{Shaded}
\begin{Highlighting}[]
\CommentTok{\#can replace "x" with student2 and 3}
\NormalTok{x }\OtherTok{\textless{}{-}}\NormalTok{ student2}
\NormalTok{x[}\FunctionTok{is.na}\NormalTok{(x)] }\OtherTok{\textless{}{-}} \DecValTok{0}
\FunctionTok{mean}\NormalTok{(x[}\SpecialCharTok{{-}}\FunctionTok{which.min}\NormalTok{(x)])}
\end{Highlighting}
\end{Shaded}

\begin{verbatim}
[1] 91
\end{verbatim}

\begin{Shaded}
\begin{Highlighting}[]
\NormalTok{x }\OtherTok{\textless{}{-}}\NormalTok{ student3}
\NormalTok{x[}\FunctionTok{is.na}\NormalTok{(x)] }\OtherTok{\textless{}{-}} \DecValTok{0}
\FunctionTok{mean}\NormalTok{(x[}\SpecialCharTok{{-}}\FunctionTok{which.min}\NormalTok{(x)])}
\end{Highlighting}
\end{Shaded}

\begin{verbatim}
[1] 12.85714
\end{verbatim}

\begin{Shaded}
\begin{Highlighting}[]
\NormalTok{grade }\OtherTok{\textless{}{-}} \ControlFlowTok{function}\NormalTok{(x) \{}
\NormalTok{  x[ }\FunctionTok{is.na}\NormalTok{(x)] }\OtherTok{\textless{}{-}} \DecValTok{0}
  \FunctionTok{mean}\NormalTok{(x[}\SpecialCharTok{{-}}\FunctionTok{which.min}\NormalTok{(x)])}
\NormalTok{\}}
\end{Highlighting}
\end{Shaded}

Now use that to grade student1 and others

\begin{Shaded}
\begin{Highlighting}[]
\FunctionTok{grade}\NormalTok{(student1)}
\end{Highlighting}
\end{Shaded}

\begin{verbatim}
[1] 100
\end{verbatim}

\begin{Shaded}
\begin{Highlighting}[]
\FunctionTok{grade}\NormalTok{(student2)}
\end{Highlighting}
\end{Shaded}

\begin{verbatim}
[1] 91
\end{verbatim}

\begin{Shaded}
\begin{Highlighting}[]
\FunctionTok{grade}\NormalTok{(student3)}
\end{Highlighting}
\end{Shaded}

\begin{verbatim}
[1] 12.85714
\end{verbatim}

Another way to do above is to highlight snippet -\textgreater{} Code (at
top) -\textgreater{} Extract Function -\textgreater{} name the function
(here I named it ``grade'')

\begin{Shaded}
\begin{Highlighting}[]
\NormalTok{grade }\OtherTok{\textless{}{-}} \ControlFlowTok{function}\NormalTok{(x) \{}
\NormalTok{  x[}\FunctionTok{is.na}\NormalTok{(x)] }\OtherTok{\textless{}{-}} \DecValTok{0}
  \FunctionTok{mean}\NormalTok{(x[}\SpecialCharTok{{-}}\FunctionTok{which.min}\NormalTok{(x)])}
\NormalTok{\}}
\end{Highlighting}
\end{Shaded}

\hypertarget{q2}{%
\section{\texorpdfstring{\textbf{Q2}}{Q2}}\label{q2}}

Who is the top scoring student overall in the gradebook?

(can get the data this way or the way I used above with the
``read.csv(''file\_name'') if it's downloaded already)

\begin{Shaded}
\begin{Highlighting}[]
\NormalTok{gradebook }\OtherTok{\textless{}{-}} \FunctionTok{read.csv}\NormalTok{(}\StringTok{"https://tinyurl.com/gradeinput"}\NormalTok{,}
\AttributeTok{row.name =} \DecValTok{1}\NormalTok{)}
\FunctionTok{head}\NormalTok{(gradebook)}
\end{Highlighting}
\end{Shaded}

\begin{verbatim}
          hw1 hw2 hw3 hw4 hw5
student-1 100  73 100  88  79
student-2  85  64  78  89  78
student-3  83  69  77 100  77
student-4  88  NA  73 100  76
student-5  88 100  75  86  79
student-6  89  78 100  89  77
\end{verbatim}

Now I want to introduce the \texttt{apply()} function.

\begin{Shaded}
\begin{Highlighting}[]
\NormalTok{results }\OtherTok{\textless{}{-}} \FunctionTok{apply}\NormalTok{(gradebook, }\DecValTok{1}\NormalTok{, grade)}
\NormalTok{results}
\end{Highlighting}
\end{Shaded}

\begin{verbatim}
 student-1  student-2  student-3  student-4  student-5  student-6  student-7 
     91.75      82.50      84.25      84.25      88.25      89.00      94.00 
 student-8  student-9 student-10 student-11 student-12 student-13 student-14 
     93.75      87.75      79.00      86.00      91.75      92.25      87.75 
student-15 student-16 student-17 student-18 student-19 student-20 
     78.75      89.50      88.00      94.50      82.75      82.75 
\end{verbatim}

I can use \texttt{which.max} to find where the largest/ max value is in
this results vector

\begin{Shaded}
\begin{Highlighting}[]
\FunctionTok{which.max}\NormalTok{(results)}
\end{Highlighting}
\end{Shaded}

\begin{verbatim}
student-18 
        18 
\end{verbatim}

\hypertarget{q3}{%
\section{\texorpdfstring{\textbf{Q3}}{Q3}}\label{q3}}

From your analysis of the gradebook, which homework was toughest on
students (i.e.~obtained the lowest scores overall?

We can use \texttt{apply()} again, but this time over columns (use 2
instead of 1 so margin=2)

\begin{Shaded}
\begin{Highlighting}[]
\FunctionTok{apply}\NormalTok{(gradebook, }\DecValTok{2}\NormalTok{, sum, }\AttributeTok{na.rm=} \ConstantTok{TRUE}\NormalTok{)}
\end{Highlighting}
\end{Shaded}

\begin{verbatim}
 hw1  hw2  hw3  hw4  hw5 
1780 1456 1616 1703 1585 
\end{verbatim}

\begin{Shaded}
\begin{Highlighting}[]
\NormalTok{lowest\_score }\OtherTok{\textless{}{-}} \FunctionTok{apply}\NormalTok{(gradebook, }\DecValTok{2}\NormalTok{, sum, }\AttributeTok{na.rm=} \ConstantTok{TRUE}\NormalTok{)}
\NormalTok{lowest\_score}
\end{Highlighting}
\end{Shaded}

\begin{verbatim}
 hw1  hw2  hw3  hw4  hw5 
1780 1456 1616 1703 1585 
\end{verbatim}

I can use my eyeballs to see that homework 2 was the toughest, but I can
also get R to tell me explicitly (incase datasets are too big in the
future)

\begin{Shaded}
\begin{Highlighting}[]
\FunctionTok{which.min}\NormalTok{(lowest\_score)}
\end{Highlighting}
\end{Shaded}

\begin{verbatim}
hw2 
  2 
\end{verbatim}

\hypertarget{q4}{%
\subsection{\texorpdfstring{\textbf{Q4}}{Q4}}\label{q4}}

Optional Extension: From your analysis of the gradebook, which homework
was most predictive of overall score (i.e.~highest correlation with
average grade score)?

\begin{Shaded}
\begin{Highlighting}[]
\CommentTok{\#cor(gradebook$hw1, results)}
\CommentTok{\#cor(gradebook$hw2, results)}

\NormalTok{mask }\OtherTok{\textless{}{-}}\NormalTok{ gradebook}
\NormalTok{mask[ }\FunctionTok{is.na}\NormalTok{(mask)] }\OtherTok{\textless{}{-}} \DecValTok{0}
\NormalTok{mask}
\end{Highlighting}
\end{Shaded}

\begin{verbatim}
           hw1 hw2 hw3 hw4 hw5
student-1  100  73 100  88  79
student-2   85  64  78  89  78
student-3   83  69  77 100  77
student-4   88   0  73 100  76
student-5   88 100  75  86  79
student-6   89  78 100  89  77
student-7   89 100  74  87 100
student-8   89 100  76  86 100
student-9   86 100  77  88  77
student-10  89  72  79   0  76
student-11  82  66  78  84 100
student-12 100  70  75  92 100
student-13  89 100  76 100  80
student-14  85 100  77  89  76
student-15  85  65  76  89   0
student-16  92 100  74  89  77
student-17  88  63 100  86  78
student-18  91   0 100  87 100
student-19  91  68  75  86  79
student-20  91  68  76  88  76
\end{verbatim}

\begin{Shaded}
\begin{Highlighting}[]
\FunctionTok{cor}\NormalTok{(mask}\SpecialCharTok{$}\NormalTok{hw5, results)}
\end{Highlighting}
\end{Shaded}

\begin{verbatim}
[1] 0.6325982
\end{verbatim}

It looks like homework 5 is highly correlated, but let's use the
\texttt{apply()} function over the masked gradebook so we don't have to
retype hw1, hw2, etc

\begin{Shaded}
\begin{Highlighting}[]
\FunctionTok{apply}\NormalTok{(mask, }\DecValTok{2}\NormalTok{, cor, }\AttributeTok{y=}\NormalTok{results)}
\end{Highlighting}
\end{Shaded}

\begin{verbatim}
      hw1       hw2       hw3       hw4       hw5 
0.4250204 0.1767780 0.3042561 0.3810884 0.6325982 
\end{verbatim}



\end{document}
